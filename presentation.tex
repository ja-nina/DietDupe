\documentclass{beamer}
\usetheme{Copenhagen}

% Basic packages
\usepackage{subfiles}
\usepackage[quiet]{fontspec}
\usepackage{geometry}
\usepackage{hyperref}
\usepackage{fancyhdr, lastpage}
\usepackage{multirow, multicol}
\usepackage{longtable}
\usepackage{siunitx}
\usepackage{float}
\usepackage{graphicx}
\usepackage{subcaption}

% Hyperref config
\hypersetup{colorlinks=true, urlcolor=blue, linkcolor=white}
\urlstyle{same}

\title{DietDupe}
\subtitle{The flavourful Feat of Finding Fantastic Food Facsimiles}
\author{Piotr Kaszubski \and Antoni Solarski \and Nina Żukowska}
\institute{Poznań University of Technology}
\date{Monday, November  6, 2023}

\begin{document}
\maketitle
\begin{frame}
	\frametitle{Hypothesis}
	By leveraging FlavorGraph, we can identify suitable
	substitutes for various dietary preferences, such as vegan,
	keto, and low-carb, while maintaining sensory satisfaction,
	ultimately enhancing the adherence and satisfaction of
	individuals following these diets.

	\vspace{1cm}
	\noindent
	By using the output of our DietDupe module we could inject it
	into the reverse-cooking pipeline and make it produce
	sensible recipes.
\end{frame}

\begin{frame}
	\frametitle{Steps}
	\begin{enumerate}
		\item Research
		\item Finding a database of food categories (example: vegan, paleo,
			keto, low carb, etc.)
		\item Provide a Mapping to FlavourGraph
		\item EXPERIMENT with approaches
		\item Pick best approach
	\end{enumerate}
\end{frame}

\begin{frame}
	\frametitle{Literature}
	\begin{itemize}
		\item FlavourGraph \url{https://github.com/lamypark/FlavorGraph}
		\item Inverse Cooking \url{https://github.com/facebookresearch/inversecooking}
	\end{itemize}

\end{frame}
\begin{frame}
	\frametitle{Datasets}
	\begin{itemize}
		\item Recipe 1M,
		\item FlavourGraph,
		\item dataset of food classification by food category (or nutriscore)
	\end{itemize}
\end{frame}
\end{document}

